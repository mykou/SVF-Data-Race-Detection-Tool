% !TEX root =  ThreadReport.tex

\section{Thread Leak and Its Case Study}
Threaded programs offer potential performance gains compared
with non-threaded programs.
A pthread is created either as joinable (by default) or detached. 
If a thread \emph{t} is created as joinable, an explicit \emph{pthread\_join}
call needs to be called by another thread (including main thread) to terminate
\emph{t} and allow the system to reclaim the memory allocated for \emph{t}. 
On the other hand, the system recycles underlying resources of a detached thread
automatically after the thread terminates.

The advantage 
\begin{itemize}
\item programs can be carried out in parallel,
\item using a thread group over using a process group is that context
switching between threads is much faster then context switching between
processes
\end{itemize}

\begin{definition}
If a thread is created as joinable but forget to be joined, then its resources
or private memory are always kept in the process space and never reclaimed.
\end{definition}

The \emph{pthread\_join} or \emph{pthread\_detach} function should eventually be
called for every thread that is created with the detachstate attribute
so that storage associated with the thread may be
reclaimed.