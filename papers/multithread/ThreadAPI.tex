% !TEX root =  ThreadReport.tex

\section{Thread API}
\subsection{Thread Management}
The following APIs create/destroy/join/detach a thread
\subsubsection{Create and Terminating a Thread:}

\begin{itemize}
\item \texttt{pthread\_create(thread, attr, start\_routine, arg)}

A thread (including main) spawns a new thread to call a \emph{start\_routine}
with an argument \emph{arg}

\item \texttt{pthread\_exit (status)}
A thread returns normally from its starting rountine, or call pthread\_exit to
early terminte itself.

\item \texttt{exec() or exit()}
The entire process is terminated
 
\end{itemize}


\subsubsection{Hard to be statically modeled:} 
The following APIs are hard to model due to its

\begin{itemize}
  \item \texttt{pthread\_cancel(thread)}

  One thread is terminated by other thread.
 
  \item \texttt{pthread\_join(threadid,status)}
 
  Blocks the calling thread until the specified \emph{threadid} thread (must be
  created joinable) terminates. Clean up any resources associated with the
  thread.

  \item \texttt{pthread\_detach(threadid)}
  The thread are never going to join with the spawner thread again
  This allows the pthread library to know whether it can immediately
  dispose of the thread resources once the thread exits (the detached case) or 
  whether it must keep them around because it may later call pthread\_join on
  the spawner thread.
  
  Multiple pthread\_detach calls on the same target thread causes
  an error
 
\end{itemize}


\subsection{Mutexes}
\subsubsection{Creating and Destroying Mutexes}
A mutex variable and its attributes need to be initialized and destroyed when
they are not used.
\begin{itemize}

  \item \texttt{pthread\_mutex\_init (mutex,attr)}

  A mutex variable must be initialized before they can be used

  \item \texttt{pthread\_mutex\_destroy (mutex)}

  If a mutex variable is destroyed, then it can not be used later

  \item \texttt{pthread\_mutexattr\_init (attr)}

  A mutex variable attribute must be initialized before they can be used

  \item \texttt{pthread\_mutexattr\_destroyed (attr)}

  If a mutex attribute is destroyed, then it can not be used later
  
\end{itemize}

\subsubsection{Locking and Unlocking Mutexes}
Locking and unlocking a mutex variable is used to protect the code segments to
avoid data race.

\begin{itemize}
  \item \texttt{pthread\_mutex\_lock (mutex)}
  
  Acquire a lock on a mutex variable by a thread, if the mutex is already
  locked by another thread, the call will be blocked until the mutex variable is released.
  
  \item \texttt{pthread\_mutex\_trylock (mutex)}
  Attempt to lock a mutex variable, if the mutex is already locked, return a
  ``busy'' to prevent deadlock 
  
  \item \texttt{pthread\_mutex\_unlock (mutex)}
  Release a mutex by the owning thread. An error occur if (1) mutex was already
  unlocked (2) the mutex is owned by another thread.
  
\end{itemize}


\subsection{Condition Variables}
\subsubsection{Creating and Destroying Condition Variables}
Similar as mutex variables, conditional variables must be initialized first and
destroyed when they are not used any more.

\begin{itemize}
  \item \texttt{pthread\_cond\_init (condition, attr)}
  
  Create a condition variable

  \item \texttt{pthread\_cond\_destroy (condition, attr)}
  
  Destroy a condition variable
   
  \item \texttt{pthread\_cond\_init (attr)}
  
  Initialize a condition variable attribute
   
  \item \texttt{pthread\_cond\_destroy (attr)}
  
  Destroy a condition variable attribute 
\end{itemize}

\subsubsection{Waiting and Signaling}
Condition variable provide another way for threads to synchronize. It
synchronize based on actual values of data instead of
controling thread access to data
\begin{itemize}
  \item \texttt{pthread\_cond\_wait (condition,mutex)}
  
  Block the calling thread until the \emph{condition} is signalled.
  It automatically and atomically releases mutex and wait for the signal.
  After signal is received the mutex will be automatically and atomically
  locked 
  
  \item  \texttt{pthread\_cond\_signal (condition)}
  
  Signal (weak up) another thread which is waiting on the condition variable.
  It should be called after \emph{mutex} is locked and unlock \emph{mutex} in
  order for \texttt{pthread\_cond\_wait} to complete
  
  \item \texttt{pthread\_cond\_broadcast (condition)}
  
  Broadcast signal to weak up more than one thread 
\end{itemize}




% \begin{tabular}{llllllll}
% & swaptions & dedup &libjpeg & libxml2 & libtool & zlib & glib & vips & yasm
% & x264 & gsl
% \end{tabular}


%\bibliographystyle{finplain}
\bibliographystyle{plain}
%\bibliography{}
